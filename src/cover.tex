\documentclass[10pt]{developercv}

\begin{document}

\begin{minipage}[t]{0.6\textwidth}
  \vspace{-\baselineskip} % Required for vertically aligning minipages

  {\HUGE\textbf{\MakeUppercase{Fagner Cunha}}}
  \vspace{5pt}

  {\huge Ph.D. Student} % Career or current job title

  \icon{MapMarker}{8}{Manaus, AM, Brazil}\\

\end{minipage}
\begin{minipage}[t]{0.3\textwidth}
  \vspace{-\baselineskip} % Required for vertically aligning minipages

  \icon{Phone}{8}{+55 92 98176 7994}\\
\icon{At}{8}{\href{mailto:fagner.cunha@icomp.ufam.edu.br}{
fagner.cunha@icomp.ufam.edu.br } } \\
\icon{Github}{8}{\href{https://github.com/alcunha}{github.com/alcunha/}}
\\ \icon{Linkedin}{8}{\href{https://linkedin.com/in/cunhaf}{
linkedin.com/in/cunhaf}}\\
\end{minipage}

\vspace{1cm}


\begin{minipage}[t]{\textwidth}
  % \vspace{-\baselineskip} % Required for vertically aligning minipages
\large
Jan 21\textsuperscript{st}, 2022\\

Dear André-Philippe Drapeau Picard,\\

I'm writing this letter to apply for the IA internship to work with computer 
vision in a project for butterfly identification, which I read on Professor 
David Rolnick's Twitter. I am a Ph.D. student at the Federal University of 
Amazonas, and my research focuses on computer vision for information extraction 
in images captured by camera traps. I'm confident my research experience makes 
me a great fit for this internship.\\

In recent years I have applied in my research several techniques to deal with 
fine-grained image classification and large-scale long-tailed datasets. During 
my master's research, I was invited to join the Project Providence, which was
revolutionary research to develop technology to monitor the Amazon biodiversity. 
In this project, I was responsible for developing deep learning models for 
animal species recognition on the camera vision module.\\

To better assess my skills, I participated in the last two editions of the 
iWildCam competition from the Workshop on Fine-Grained Visual Categorization 
(FGVC/CVPR). In 2020, I was in the fourth position of iWildCam whose 
task was to identify the animal species in camera trap images. In 2021 I won 
iWildCam's challenge, where we had to count the number of animals of each 
species present in a sequence of images. Additionally, in 2021 I participated in 
the iNaturalist competition, where my solution included an image classifier for 
the 10 thousand species and a geo prior model, being ranked in the 6th 
position. Solutions are available on my GitHub.\\

Thus, relying on my experience with deep learning for camera trap images 
classification and my experiences in competitions dealing with the same problem 
class, I believe I have the required profile and skills to contribute to your 
project's success.\\

Please don’t hesitate to reach out if you have any questions about my 
background. Thank you for your time and I’m looking forward to attending this 
project.\\\\

Sincerely,\\

Fagner Cunha

\end{minipage}

\end{document}
