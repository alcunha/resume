\documentclass[10pt]{developercv}

\begin{document}

\begin{minipage}[t]{0.6\textwidth}
  \vspace{-\baselineskip} % Required for vertically aligning minipages

  {\HUGE\textbf{\MakeUppercase{Fagner Cunha}}}
  \vspace{5pt}

  {\huge Ph.D. Student} % Career or current job title

  \icon{MapMarker}{8}{Manaus, AM, Brazil}\\

\end{minipage}
\begin{minipage}[t]{0.3\textwidth}
  \vspace{-\baselineskip} % Required for vertically aligning minipages

  \icon{Phone}{8}{+55 92 98176 7994}\\
\icon{At}{8}{\href{mailto:fagner.cunha@icomp.ufam.edu.br}{
fagner.cunha@icomp.ufam.edu.br } } \\
\icon{Github}{8}{\href{https://github.com/alcunha}{github.com/alcunha/}}
\\ \icon{Linkedin}{8}{\href{https://linkedin.com/in/cunhaf}{
linkedin.com/in/cunhaf}}\\
\end{minipage}

\vspace{1cm}


\begin{minipage}[t]{\textwidth}
  % \vspace{-\baselineskip} % Required for vertically aligning minipages
\large
Jan 21\textsuperscript{st}, 2022\\

Dear André-Philippe Drapeau Picard,\\

I'm writing to apply for the IA internship to work on computer vision for 
butterfly identification that I saw on Professor David Rolnick's Twitter. I am a 
Ph.D. student at the Federal University of Amazonas and my research focuses on 
computer vision for extracting information from camera traps images. I'm 
confident my research experience makes me a great fit for this internship.\\

Nos últimos anos tenho aplicado durante a minha pesquisa diversas técnicas para 
lidar tanto com fine-grained image classification quanto para long tail 
datasets. During my master's, I was invited to join the Project 
Providence, a revolutionary project to develop technology to monitor the Amazon 
biodiversity. In this project, I was responsible for developing deep learning 
models for animal species recognition on the camera vision module.\\

Para avaliar melhor minhas habilidades, participei das duas últimas edições da 
competição iWildCam do Workshop on Fine-Grained Visual Categorization (CVPR). 
No ano de 2020 eu fiquei na quarta posição do iWildCam cuja tarefa era a 
classificação de espécies de animais em imagens de armadilhas fotográficas. Em 
2021 eu venci o desafio do iWildCam que era contar quantos indivíduos de cada 
espécie estavam presentes em cada sequência de sequência de imagens. 
Adicionalmente, em 2021 participei da competição do iNaturalist, onde minha 
solução incluía além do classificador de imagens para as 10 mil espécies, um 
modelo de geo prior, tendo minha solução ficado na 6a posição. As soluções 
estão disponíveis no meu github.\\

Dessa forma, contando com minha experiência com classificação de imagens de 
armadilhas fotográficas e minha experiências nas competições que tratam da 
mesma classe de problema, eu acredito que tenho o perfil requisitado e as 
skills necessárias para contribuir para o sucesso do projeto.\\

Please don't hesitate to reach out if you have any questions about my 
background. Thank you for your time and I'm looking forward to hearing from 
you.\\\\

Sincerely,\\

Fagner Cunha

\end{minipage}

\end{document}
