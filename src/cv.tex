\documentclass[10pt]{developercv} % Default font size, values from 8-12pt are recommended

\begin{document}

\pagestyle{plain}

\begin{minipage}[t]{0.6\textwidth}
  \vspace{-\baselineskip} % Required for vertically aligning minipages

  {\HUGE\textbf{\MakeUppercase{Fagner Cunha}}}
  \vspace{5pt}

  \icon{MapMarker}{8}{Vargem Grande Paulista, SP, Brazil}\\

\end{minipage}
\begin{minipage}[t]{0.3\textwidth}
  \vspace{-\baselineskip} % Required for vertically aligning minipages

%   \icon{Phone}{8}{+55 92 98176 7994}\\
\icon{At}{8}{\href{mailto:fagner.cunha99@gmail.com}{
fagner.cunha99@gmail.com } } \\
\icon{Github}{8}{\href{https://github.com/alcunha}{github.com/alcunha/}}
\\ \icon{Linkedin}{8}{\href{https://linkedin.com/in/cunhaf}{
linkedin.com/in/cunhaf}}\\
\end{minipage}

\vspace{0.5cm}

%-------------------------------------------------------------------------------
%	EDUCATION
%-------------------------------------------------------------------------------

\cvsect{Education}

\begin{entrylist}
  \entry
  {2019 -- Present\\\footnotesize{Expected 12/2025}}
  {Ph.D. in Informatics (in progress)}
  {Federal University of Amazonas}
  {Advisor: Eulanda Miranda\\Co-Advisor: Juan Colonna\\
  This work improves camera-trap image analysis with deep learning, addressing
species identification, counting, and empty-frame filtering under real-world
challenges like long-tail data and domain shift.
  }

  \entry
  {2017 -- 2019}
  {M.Sc. in Informatics}
  {Federal University of Amazonas}
  {Advisor: Eulanda Miranda\\Co-Advisor: Juan Colonna\\
  Thesis title: Um estudo sobre abordagens para avaliação out-of-sample de
modelos de classificação de animais em imagens de armadilhas fotográficas}

  \entry
  {2009 -- 2015}
  {B.Sc. in Computer Engineering}
  {Federal University of Amazonas}
  {Embedded Software Engineering, Machine Learning}

 \entry
  {2006 -- 2008}
  {Technician in Electronics}
  {\parbox[t]{6.5cm}{\raggedleft Federal Institute of Education, Science \\ and
Technology of Amazonas}}
  {Technical High School}
\end{entrylist}


%-------------------------------------------------------------------------------
%	EXPERIENCE
%-------------------------------------------------------------------------------

\cvsect{Experience}

\begin{entrylist}

\entry{2022 -- 2024}
  {Collaborating Researcher}
  {Mila - Quebec Artificial Intelligence Institute}
  {Research and development of CV models for insect identification; Rolnick
Lab.}


\entry{2022 -- 2024}
  {Artificial Intelligence Intern}
  {eButterfly}
  {Model development for butterfly species identification, with dataset curation
and integration of geo-priors for fine-grained classification.}

  \entry
  {2021 -- 2023}
  {Undergraduate Research Mentor}
  {IComp - UFAM}
  {Mentoring of undergraduate students in the Samsung UFAM Project for Education
and Research (SUPER).}

\entry{2017 -- 2018}
  {Computer Vision Research Assistant}
  {Mamiraua Institute}
  {CV researcher on Project Providence: developed and optimized animal species
classifiers for embedded camera-trap systems.}


  \entry{2015 -- 2017}
  {Computer Engineer}
  {IATECAM}
  {Embedded software engineer with focus on Altera FPGA systems: hardware design
with Qsys, Linux driver/app development.}

  \entry{2013 -- 2015}
  {Software Developer}
  {CETELI - UFAM}
  {Front-end Android development for Samsung R\&D (SIDIA)}

\entry{2012 -- 2013}
  {Embedded Software Engineer Intern}
  {Map Innovation}
  {Developed industrial data collection systems}

\entry{2008 -- 2008}
  {Electronic Engineering Intern}
  {Terra da Amazonia Ltda}
  {Supported motherboard production as manufacturing test engineer}
\end{entrylist}

\vspace{1.2cm}

\cvsect{Honors \& Awards}

\begin{entrylist}
  \entry
  {2021}
  {1st Place in the iWildCam 2021 Challenge}
  {CVPR - FGVR Workshop}
  {Count the number of animals of each species present in a sequence of images.}

  \entry
  {2012}
  {Best Paper Award at the Workshop of Undergraduate Research}
  {SBSEG}
  {Detecção de Phishing em Páginas Web Utilizando Técnicas de Aprendizagem de
Máquina.}

  \entry
  {2011}
  {Professor Abraham Moysés Cohen Award}
  {Federal University of Amazonas}
  {Best undergraduate research work in Exact Sciences at the XIX Congress of
Scientific Initiation for the work: Detecção de Phishing em Páginas Web.}

\entry
  {2006}
  {Honorable Mention}
  {Brazilian Public Schools Mathematics Olympiad}
  {Honorable Mention – Level 3 (High School)}

\entry
  {2005}
  {Silver Medal}
  {Brazilian Public Schools Mathematics Olympiad}
  {Silver Medal - Level 2 (Grades 5 to 8)}


\end{entrylist}


\cvsect{Publications}
($^\ast$ denotes equal contribution)
\vspace{0.3cm}

\textbf{Journal Publications}

\begin{enumerate}
 \item Roy, D. B., Alison, J., August, T. A., Bélisle, M., Bjerge, K., Bowden,
J. J., Bunsen, M. J., \textbf{Cunha, F.}, Geissmann, Q., Goldmann, K.,
Gomez-Segura, A., Jain, A., Huijbers, C., Larrivée, M., Lawson, J. L., Mann, H.
M., Mazerolle, M. J., McFarland, K. P., Pasi, L., Peters, S., Pinoy, N.,
Rolnick, D., Skinner, G. L., Strickson, O. T., Svenning, A., Teagle, S., \&
Høye, T. T. (2024). Towards a standardized framework for AI-assisted,
image-based monitoring of nocturnal insects. Philosophical Transactions of the
Royal Society B, 379(1904), 20230108.

 \item \textbf{Cunha, F.}, dos Santos, E. M., \& Colonna, J. G. (2023). Bag of
tricks for long-tail visual recognition of animal species in camera-trap
images. Ecological Informatics, 76, 102060.

 \item dos Santos, E. M., \textbf{Cunha, F.}, Colonna, J. G., \& Carvalho, J.
R. (2023). Monitoramento Ambiental Não Invasivo Utilizando Dados de Sensores e
Técnicas de Aprendizagem de Máquina. Computação Brasil, (50), 24-28.\\

\end{enumerate}

\textbf{Conference Publications}

\begin{enumerate}
 \item Jain, A.$^\ast$, \textbf{Cunha, F.$^\ast$}, Bunsen, M. J.$^\ast$, Cañas,
J. S., Pasi, L., Pinoy, N., Helsing, F., Russo, J., Botham, M., Sabourin, M.,
Fréchette, J., Anctil, A., Lopez, Y., Navarro, E., Perez Pimentel, F., Zamora,
A. C., Ramirez Silva, J. A., Gagnon, J., August, T., Bjerge, K., Gomez Segura,
A., Bélisle, M., Basset, Y., McFarland, K. P., Roy, D., Høye, T. T., Larrivée,
M., \& Rolnick, D. (2024). Insect identification in the wild: The AMI dataset.
In European Conference on Computer Vision (pp. 55–73). Cham: Springer Nature
Switzerland.

 \item Alencar, L., \textbf{Cunha, F.}, \& Dos Santos, E. M. (2024). Zero and
Few-Shot Learning with Modern MLLMs to Filter Empty Images in Camera Trap Data.
In 2024 37th SIBGRAPI Conference on Graphics, Patterns and Images (SIBGRAPI)
(pp. 1-6). IEEE.

 \item Alencar, L., \textbf{Cunha, F.}, \& dos Santos, E. M. (2023). A
context-aware approach for filtering empty images in camera trap data using
siamese network. In 2023 36th SIBGRAPI Conference on Graphics, Patterns and
Images (SIBGRAPI) (pp. 85-90). IEEE.

 \item Fonseca, V. L., \textbf{Cunha, F.}, Andrade, L., Colonna, J. G., \& De
Yong, D. (2023). Classification of Tropical Disease-carrying Mosquitoes Using
Deep Learning and SHAP. In Simpósio Brasileiro de Computação Aplicada à Saúde
(SBCAS) (pp. 25-34). SBC.

 \item Medeiros, V. P., \textbf{Cunha, F.}, dos Santos, E. M., \& Souto, E.
(2022). Um Modelo de Aprendizado Profundo Multimodal para Classificação de
Estresse Utilizando Sinais Obtidos por Dispositivos Vestíveis de Pulso. In
Simpósio Brasileiro de Computação Aplicada à Saúde (SBCAS) (pp. 370-380). SBC.

 \item Queiroz, D., \textbf{Cunha, F.}, Souza, L. R., \& Colonna, J. G. (2022).
Investigando a relação entre os aminoácidos de proteínas do vírus da dengue e o
desfecho clínico do paciente. In Simpósio Brasileiro de Computação Aplicada à
Saúde (SBCAS) (pp. 92-97). SBC.

 \item Jacinto, L. M., Chateaubriand, A. D., Evangelista, M. R., Schultz, V. P.,
Jacinto, O. M., \textbf{do Rego Cunha, F. F.}, \& Rodrigues, D. L. (2013).
Estudos para o desenvolvimento de um ambiente saudável em uma universidade
amazônica: A experiência de alunos de engenharia e design. In Proceedings of the
Safety, Health and Environment World Congress (Vol. 13, pp. 304–308). \\
\end{enumerate}


\textbf{Preprints and Workshop Papers}

\begin{enumerate}
 \item Jain, A.$^\ast$, \textbf{Cunha, F.$^\ast$}, Bunsen, M.$^\ast$, Pasi, L.,
Viklund, A., Larrivée, M., \& Rolnick, D. (2023). A machine learning pipeline
for automated insect monitoring. NeurIPS 2023 Workshop on Tackling Climate
Change with Machine Learning. arXiv preprint arXiv:2406.13031.

 \item \textbf{Cunha, F.}, dos Santos, E. M., Barreto, R., \& Colonna, J. G.
(2021). Filtering empty camera trap images in embedded systems. In Proceedings
of the IEEE/CVF Conference on Computer Vision and Pattern Recognition (CVPR)
Workshops (pp. 2438-2446).

 \item \textbf{Cunha, F. F. R.}, dos Santos, E. M., \& Souto, E. (2012).
Detecção de phishing em páginas web utilizando técnicas de aprendizagem de
máquina. In Proceedings of the IV Workshop de Trabalhos de Iniciação Científica
e de Graduação, Anais do XII Simpósio Brasileiro em Segurança da Informação e
de Sistemas Computacionais (pp. 491–500). Curitiba, Brazil. \\
\end{enumerate}


%----------------------------------------------------------------------------------------
%	ADDITIONAL INFORMATION
%----------------------------------------------------------------------------------------

\begin{minipage}[t]{0.3\textwidth}
  \vspace{-\baselineskip} % Required for vertically aligning minipages

  \cvsect{Languages}

  \begin{skills}
    \skillset{Portuguese}{Native speaker}
    \skillset{English}{Professional working proficiency}
  \end{skills}
\end{minipage}

\end{document}
