\documentclass[10pt]{developercv}

\begin{document}

\begin{minipage}[t]{0.6\textwidth}
  \vspace{-\baselineskip}

  {\HUGE\textbf{\MakeUppercase{Fagner Cunha}}}
  \vspace{5pt}

  {\Large Pesquisador em Visão Computacional}

  \icon{MapMarker}{8}{Vargem Grande Paulista, SP, Brasil}\\

\end{minipage}
\begin{minipage}[t]{0.3\textwidth}
  \vspace{-\baselineskip}

\icon{At}{8}{\href{mailto:fagner.cunha99@gmail.com}{
fagner.cunha99@gmail.com } } \\
\icon{Github}{8}{\href{https://github.com/alcunha}{github.com/alcunha/}}
\\ \icon{Linkedin}{8}{\href{https://linkedin.com/in/cunhaf}{
linkedin.com/in/cunhaf}}\\
\end{minipage}

\vspace{0.5cm}

\cvsect{Quem Sou Eu?}

\begin{minipage}[t]{\textwidth}

Profissional de Inteligência Artificial e Aprendizado de Máquina com experiência
em visão computacional e deep learning aplicados a dados complexos. Atuei em
projetos internacionais (Mila/Canadá, eButterfly) e nacionais (Instituto
Mamirauá), desenvolvendo modelos robustos para reconhecimento de espécies e
monitoramento ambiental. Experiência em preparação e curadoria de dados,
treinamento, customização e otimização de arquiteturas como Vision Transformers,
EfficientNets e MobileNets, além de adaptação de modelos para cenários do mundo
real. Atualmente curso doutorado na UFAM, com pesquisa em visão computacional
aplicada a monitoramento ecológico. \\

\end{minipage}

\cvsect{Experiência Profissional}

\begin{entrylist}

\entry{3/2022 -- 3/2024\\\footnotesize{Remoto}}
  {Pesquisador Colaborador}
  {Mila - Quebec Artificial Intelligence Institute}
  {
   Colaborei no desenvolvimento de modelos para identificação automática de
insetos em imagens. Principais contribuições:\\
    \begin{contributionlist}
      \contribution{Criação, curadoria e publicação de um dataset de referência
(The AMI Dataset: Insect Identification in the Wild}
      \contribution{Treinamento de modelos para identificar espécies e estágios
de vida de insetos em cenários complexos e desbalanceados}
      \contribution{Abordei desafios de mudança de domínio com modelos treinados
em imagens indexadas pelo GBIF para dados reais de armadilhas fotográficas}
    \end{contributionlist}\\
\texttt{Visão Computacional}\slashsep\texttt{Domain
Shift}\slashsep\texttt{Long-tail} \slashsep\texttt {PyTorch}}\\

\entry{3/2022 -- 3/2024\\\footnotesize{Remoto}}
  {Pesquisador em Inteligência Artificial}
  {eButterfly}
  {
  Responsável pelo desenvolvimento de um modelo global de reconhecimento de
borboletas para a plataforma eButterfly (projeto internacional de ciência
cidadã). Contribuições principais:\\
  \begin{contributionlist}
      \contribution{Curadoria de uma base de dados em larga escala cobrindo mais
de 18.000 espécies de borboletas utilizando dados indexados pelo GBIF}
      \contribution{Melhoria do desempenho do modelo com uso de informações
biológicas e distribuição geográfica}
      \contribution{Treinamento e otimização para cenários de classificação
fine-grained sob distribuição altamente desbalanceada de cauda longa}
    \end{contributionlist}\\
\texttt{Visão Computacional}\slashsep\texttt{Classificação Fine-grained}
\slashsep\texttt {PyTorch}}\\

\entry{4/2017 -- 6/2018\\\footnotesize{Manaus, AM, Brasil}}
  {Pesquisador em Visão Computacional}
  {Instituto Mamirauá}
  {
    Atuei no desenvolvimento de modelos embarcados para reconhecimento de
animais em armadilhas fotográficas no Projeto Providence. Principais
atividades:\\
    \begin{contributionlist}
      \contribution{Preparação e limpeza de datasets de imagens para treinamento
de modelos}
      \contribution{Treinamento e avaliação de classificadores de espécies
animais utilizando redes neurais convolucionais profundas}
      \contribution{Otimização para execução em hardware de baixo custo
(Raspberry Pi)}
    \end{contributionlist}\\
\texttt{TensorFlow}\slashsep\texttt{Visão Computacional}
\slashsep\texttt {Raspberry Pi}}\\


\entry{3/2015 -- 3/2017\\\footnotesize{Manaus, AM}}
  {Engenheiro de Computação}
  {IATECAM - Instituto Ambiental e Tecnológico da Amazônia}
  {
  Atuação como engenheiro de software embarcado em projetos com FPGA Altera.
    \begin{contributionlist}
      \contribution{Desenvolvimento de drivers Linux para IPs customizados}
      \contribution{Aplicações embarcadas em Linux e integração de hardware com
Qsys}
      \contribution{Soluções de visão computacional para detecção de defeitos de
soldagem}
    \end{contributionlist}\\
\texttt{C}\slashsep\texttt{FPGA}\slashsep\texttt{Linux Embarcado}}
\\

\entry{11/2013 -- 3/2015\\\footnotesize{Manaus, AM}}
  {Desenvolvedor de Software}
  {CETELI - UFAM (Cliente: Samsung SIDIA)}
  {
  Desenvolvimento de aplicativos Android e projetos web para P\&D da Samsung.
    \begin{contributionlist}
      \contribution{Criação de widgets customizados, animações e layouts para
Android}
      \contribution{Desenvolvimento web com PHP, HTML5, CSS3, JavaScript e
jQuery}
    \end{contributionlist}\\
\texttt{Android}\slashsep\texttt{Java}\slashsep\texttt{Web}} \\

\entry{2/2012 -- 8/2013\\\footnotesize{Manaus, AM}}
  {Estagiário em Engenharia de Software Embarcado}
  {Map Innovation}
  {
  Desenvolvimento de equipamento industrial de coleta de dados.
    \begin{contributionlist}
      \contribution{Drivers para Linux embarcado e customização do bootloader
U-Boot}
      \contribution{Firmware para ARM com FreeRTOS e suporte a boot pela rede}
      \contribution{Sistema de testes funcionais para controle de qualidade}
    \end{contributionlist}\\
\texttt{C}\slashsep\texttt{FreeRTOS}\slashsep\texttt{Linux Embarcado}}
\\

\end{entrylist}

\cvsect{Formação Acadêmica}

\begin{entrylist}
  \entry
  {4/2019 -- Atual\\\footnotesize{Previsto 12/2025 \\\\Manaus, AM, Brasil}}
  {Doutorado em Informática}
  {Universidade Federal do Amazonas}
  {Orientadora: Eulanda Miranda | Coorientador: Juan Colonna\\
  Área de pesquisa: Visão Computacional \\

  Esta pesquisa visa avançar a análise automatizada de imagens de armadilhas
fotográficas utilizando técnicas de visão computacional e aprendizado profundo.
Foca em tarefas-chave de monitoramento ecológico como filtragem de imagens
vazias, classificação de espécies (em nível de imagem e de evento de captura) e
contagem de indivíduos.
  }
  \entry
  {3/2017 -- 4/2019\\\footnotesize{Manaus, AM, Brasil}}
  {Mestrado em Informática}
  {Universidade Federal do Amazonas}
  {Orientadora: Eulanda Miranda | Coorientador: Juan Colonna\\
  Área de pesquisa: Visão Computacional\\\\
  Dissertação: Um estudo sobre abordagens para avaliação out-of-sample de
modelos de classificação de animais em imagens de armadilhas fotográficas}
  \entry
  {3/2009 -- 1/2015\\\footnotesize{Manaus, AM, Brasil}}
  {Engenharia da Computação}
  {Universidade Federal do Amazonas}
  {Ênfase em Software Embarcado e Aprendizado de Máquina}
\end{entrylist}

% \cvsect{Publicações}
%
% \begin{minipage}[t]{\textwidth}
%
% Aditya Jain*, \textbf{Fagner Cunha}*, Michael Bunsen*, Juan Sebasti{\'a}n
% Ca{\~n}as, L{\'e}onard Pasi, David Rolnick, et al. Insect identification in the
% wild: The AMI dataset. In European Conference on Computer Vision (ECCV), 2024,
% pp. 55-73.\\
% {*\scriptsize Equal contribution}\\
%
% Aditya Jain*, \textbf{Fagner Cunha}*, Michael Bunsen*, L{\'e}onard Pasi, Anna
% Viklund, Maxim Larrivée \& David Rolnick. A machine learning pipeline for
% automated insect monitoring. In NeurIPS 2023 Workshop on Tackling Climate
% Change with Machine Learning. arXiv preprint arXiv:2406.13031.\\
% {*\scriptsize Equal contribution}\\
%
% \textbf{Fagner Cunha}, Eulanda M. dos Santos, \& Juan G.
% Colonna. Bag of tricks for long-tail visual recognition of animal species in
% camera-trap images.  Ecological Informatics, v. 76, p. 102060, 2023.\\
%
% \textbf{Fagner Cunha}, Eulanda M. dos Santos, Raimundo Barreto, \& Juan G.
% Colonna. Filtering Empty Camera Trap Images in Embedded Systems. In Proceedings
% of the IEEE/CVF Conference on Computer Vision and Pattern
% Recognition (CVPR) Workshops, 2021, pp. 2438-2446.\\
%
% \end{minipage}
%
% \cvsect{Prêmios e Reconhecimentos}
%
% \begin{entrylist}
%   \entry
%   {06/2021}
%   {1º Lugar na Competição iWildCam 2021}
%   {CVPR - FGVR Workshop}
%   {Contagem do número de animais de cada espécie presente em uma sequência de
% imagens.}
%   \entry
%   {11/2012}
%   {Melhor Artigo no Workshop de Iniciação Científica}
%   {SBSEG}
%   {Prêmio recebido pelo artigo: Detecção de Phishing em Páginas Web Utilizando
% Técnicas de Aprendizagem de Máquina}
%   \entry
%   {08/2011}
%   {Prêmio Professor Abraham Moysés Cohen}
%   {Universidade Federal do Amazonas}
%   {Melhor trabalho de iniciação científica em Ciências Exatas no XIX Congresso
% de Iniciação Científica pelo trabalho: Detecção de Phishing em Páginas Web.}
% \end{entrylist}

\begin{minipage}[t]{0.3\textwidth}
  \vspace{-\baselineskip}

  \cvsect{Idiomas}

  \begin{skills}
    \skillset{Português}{nativo}
    \skillset{Inglês}{proficiência profissional}
  \end{skills}
\end{minipage}

\end{document}
