\documentclass[10pt]{developercv} % Default font size, values from 8-12pt are recommended

\begin{document}

\begin{minipage}[t]{0.6\textwidth}
  \vspace{-\baselineskip} % Required for vertically aligning minipages

  {\HUGE\textbf{\MakeUppercase{Fagner Cunha}}}
  \vspace{5pt}

  {\huge Ph.D. Student} % Career or current job title

  \icon{MapMarker}{8}{Manaus, AM, Brazil}\\

\end{minipage}
\begin{minipage}[t]{0.3\textwidth}
  \vspace{-\baselineskip} % Required for vertically aligning minipages

  \icon{Phone}{8}{+55 92 98176 7994}\\
\icon{At}{8}{\href{mailto:fagner.cunha@icomp.ufam.edu.br}{
fagner.cunha@icomp.ufam.edu.br } } \\
\icon{Github}{8}{\href{https://github.com/alcunha}{github.com/alcunha/}}
\\ \icon{Linkedin}{8}{\href{https://linkedin.com/in/cunhaf}{
linkedin.com/in/cunhaf}}\\
\end{minipage}

% \vspace{0.5cm}

\cvsect{Who Am I?}

\begin{minipage}[t]{\textwidth}
  % \vspace{-\baselineskip} % Required for vertically aligning minipages

I am a Ph.D. student at the Federal University of Amazonas (UFAM). My research 
focuses on computer vision and machine learning for extracting information from 
highly imbalanced image datasets produced by camera traps. During my Master's, I 
was responsible for developing deep learning models for animal species 
recognition on the Project Providence, a revolutionary project to develop 
technology to monitor the Amazon biodiversity. Previously, I had worked as an 
embedded software engineer in the industry.

\end{minipage}
% \hfill % Whitespace between

%-------------------------------------------------------------------------------
%	EDUCATION
%-------------------------------------------------------------------------------

\cvsect{Education}

\begin{entrylist}
  \entry
  {Expected 2023\\\footnotesize{Manaus, AM, Brazil}}
  {Ph.D. Student - Informatics}
  {Federal University of Amazonas}
  {Advisor: Eulanda Miranda | Co-Advisor: Juan Colonna\\
  Research area: Computer Vision}
  \entry
  {04/2019\\\footnotesize{Manaus, AM, Brazil}}
  {Master's Degree - Informatics}
  {Federal University of Amazonas}
  {Advisor: Eulanda Miranda | Co-Advisor: Juan Colonna\\
  Research area: Computer Vision\\\\
 Thesis: A study of approaches for out-of-sample evaluation of models for 
animal classification in camera trap images}
  \entry
  {01/2015\\\footnotesize{Manaus, AM, Brazil}}
  {Engineer's Degree - Computer Engineering}
  {Federal University of Amazonas}
  {Embedded Software Engineering, Machine Learning}
\end{entrylist}


\cvsect{Honors \& Awards}

\begin{entrylist}
  \entry
  {06/2021}
  {1st Place in the iWildCam 2021 Competition}
  {CVPR - FGVR Workshop}
  {Count the number of animals of each species present in a sequence of images.}
  \entry
  {11/2012}
  {Best Paper Award at the Workshop of Undergraduate Research}
  {SBSEG}
  {Award received for the paper: Detection 
of Phishing Webpages Using Machine Learning Techniques}
  \entry
  {08/2011}
  {Professor Abraham Moysés Cohen Award}
  {Federal University of Amazonas}
  {Best undergraduate research work in Exact Sciences at the XIX Congress of 
Scientific Initiation for the work: Detection of Phishing Webpages.}
\end{entrylist}


\cvsect{Publications}

\begin{minipage}[t]{\textwidth}
  % \vspace{-\baselineskip} % Required for vertically aligning minipages

\textbf{Fagner Cunha}, Eulanda M. dos Santos, Raimundo Barreto, \& Juan G. 
Colonna. Filtering Empty Camera Trap Images in Embedded Systems. In Proceedings 
of the IEEE/CVF Conference on Computer Vision and Pattern 
Recognition (CVPR) Workshops, 2021, pp. 2438-2446.\\

\textbf{Fagner Cunha}, Eulanda M. dos Santos, \& Eduardo J. P. Souto. Detection 
of Phishing Webpages Using Machine Learning Techniques. In Proceedings 
of the 12th Brazilian Symposium on Information and Computer System Security, 
2012, pp. 491-500.
\end{minipage}

%----------------------------------------------------------------------------------------
%	EXPERIENCE
%----------------------------------------------------------------------------------------

\cvsect{Professional Experience}

\begin{entrylist}
  \entry
  {7/2021 -- present\\\footnotesize{Manaus, AM, Brazil}}
  {Undergraduate Research Mentor}
  {IComp - UFAM}
  {Responsible for mentoring 6 undergraduate students in Computer Science 
through their first research project using computer vision and machine learning 
under the Samsung UFAM Project for Education and Research (SUPER).\\ 
\texttt{Mentoring}\slashsep\texttt{TensorFlow}\slashsep\texttt{Computer 
Vision}\slashsep\texttt{Python}}

\entry{4/2017 -- 6/2018\\\footnotesize{Manaus, AM, Brazil}}
  {Computer Vision Research Assistant}
  {Mamiraua Institute}
  {
    I worked as a computer vision researcher on Project Providence. I was 
responsible for developing deep learning models for animal species recognition 
on the camera vision module. Tasks included:\\
    \begin{contributionlist}
      \contribution{Analysis, cleaning, and preparation of the camera trap 
images dataset used for model training}
      \contribution{Training and evaluation of animal species classifiers using 
deep convolutional neural networks}
      \contribution{Optimizing the models to run on the Providence camera vision 
module, which is based on a Raspberry Pi development board}
    \end{contributionlist}\\
\texttt{TensorFlow}\slashsep\texttt{Computer Vision}
\slashsep\texttt {Raspberry Pi}\slashsep\texttt{Camera Traps}}
  
  \entry{3/2015 -- 3/2017\\\footnotesize{Manaus, AM, Brazil}}
  {Computer Engineer}
  {IATECAM}
  {
    I mainly worked as an embedded software engineer for projects using Altera 
FPGA technology. My responsibilities included:\\
    \begin{contributionlist}
      \contribution{Develop Linux device drivers for customized IP blocks}
      \contribution{Develop embedded Linux applications}
      \contribution{Develop computer vision models for detecting soldering 
problems}
    \end{contributionlist}\\
    \texttt{C}\slashsep\texttt{FPGA}\slashsep\texttt{Embedded 
Software}\slashsep\texttt{Linux}\slashsep\texttt{Computer Vision}}
  
  \entry{11/2013 -- 3/2015\\\footnotesize{Manaus, AM, Brazil}}
  {Software Developer}
  {CETELI - UFAM}
  {
    Client: Samsung R\&D Center (SIDIA)\\
    \begin{contributionlist}
      \contribution{Developed apps for Android platform with focus on front-end: 
design of custom widgets, layouts, custom animations, etc.}
      \contribution{Developed web projects using PHP, HTML5, CSS3, JavaScript, 
and JQuery.}
    \end{contributionlist}\\
    \texttt{Java}\slashsep\texttt{PHP}\slashsep\texttt{JavaScript}\slashsep\texttt{Android}}

\entry{2/2012 -- 8/2013\\\footnotesize{Manaus, AM, Brazil}}
  {Embedded Software Engineer Intern}
  {Map Innovation}
  {
    I worked on the development of an industrial data collection equipment. 
Tasks included::\\
    \begin{contributionlist}
      \contribution{Develop device drivers for embedded Linux}
      \contribution{Customize the bootloader U-Boot}
      \contribution{Develop firmware for ARM using FreeRTOS}
      \contribution{Develop a complete functional testing system}
    \end{contributionlist}\\
\texttt{C}\slashsep\texttt{Device 
Drivers}\slashsep\texttt{Embedded Software}\slashsep\texttt 
{ Embedded Linux }}

\entry{4/2008 -- 1/2009\\\footnotesize{Manaus, AM, Brazil}}
  {Electronic Engineering Intern}
  {Terra da Amazonia Ltda.}
  {I worked on the manufacturing test engineering team responsible for the 
motherboards production lines. My responsibilities included identifying and 
fixing problems on test stations and deploying and adapting functional tests for 
new products.}
\end{entrylist}

%----------------------------------------------------------------------------------------
%	ADDITIONAL INFORMATION
%----------------------------------------------------------------------------------------

\begin{minipage}[t]{0.3\textwidth}
  \vspace{-\baselineskip} % Required for vertically aligning minipages

  \cvsect{Languages}

  \begin{skills}
    \skillset{Portuguese}{native}
    \skillset{English}{professional working proficiency}
  \end{skills}
\end{minipage}

\end{document}
