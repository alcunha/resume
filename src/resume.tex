\documentclass[10pt]{developercv} % Default font size, values from 8-12pt are recommended

\begin{document}

\begin{minipage}[t]{0.6\textwidth}
  \vspace{-\baselineskip} % Required for vertically aligning minipages

  {\HUGE\textbf{\MakeUppercase{Fagner Cunha}}}
  \vspace{5pt}

  {\Large Computer Vision Researcher} % Career or current job title

  \icon{MapMarker}{8}{Vargem Grande Paulista, SP, Brazil}\\

\end{minipage}
\begin{minipage}[t]{0.3\textwidth}
  \vspace{-\baselineskip} % Required for vertically aligning minipages

%   \icon{Phone}{8}{+55 92 98176 7994}\\
\icon{At}{8}{\href{mailto:fagner.cunha99@gmail.com}{
fagner.cunha99@gmail.com } } \\
\icon{Github}{8}{\href{https://github.com/alcunha}{github.com/alcunha/}}
\\ \icon{Linkedin}{8}{\href{https://linkedin.com/in/cunhaf}{
linkedin.com/in/cunhaf}}\\
\end{minipage}

\vspace{0.5cm}

\cvsect{Who Am I?}

\begin{minipage}[t]{\textwidth}
  % \vspace{-\baselineskip} % Required for vertically aligning minipages

I am an AI/ML researcher specialized in computer vision and deep learning, with
strong experience turning research into practical solutions. I have hands-on
experience with dataset preparation and cleaning, training and customizing deep
learning architectures such as Vision Transformers, EfficientNets, and
MobileNets, as well as adapting models for real-world and resource-constrained
scenarios. Currently, I am a PhD candidate at UFAM, developing computer vision
solutions for ecological monitoring challenges under real-world constraints,
such as those posed by camera-trap data. \\

\end{minipage}
% \hfill % Whitespace between


%-------------------------------------------------------------------------------
%	EXPERIENCE
%-------------------------------------------------------------------------------

\cvsect{Professional Experience}

\begin{entrylist}

\entry{3/2022 -- 3/2024\\\footnotesize{Remote}}
  {Collaborating Researcher}
  {Mila - Quebec Artificial Intelligence Institute}
  {
  I was responsible for preparing datasets and training computer vision models
for insect identification. During my collaboration, I:\\
    \begin{contributionlist}
      \contribution{Built and curated the dataset used for model training,
including the publication of The AMI Dataset (Insect Identification in the
Wild)}
      \contribution{Trained models to classify insect life stages, distinguish
moths from other insect species, and identify moths at multiple taxonomic levels
(family, genus, species) under extreme fine-grained and long-tail distribution
conditions}
      \contribution{Tackled domain shift challenges in applying models trained
on GBIF-indexed images to real-world camera-trap data}
    \end{contributionlist}\\
\texttt{Computer Vision}\slashsep\texttt{Domain
shift}\slashsep\texttt{Long-tail} \slashsep\texttt {PyTorch}}\\

\entry{3/2022 -- 3/2024\\\footnotesize{Remote}}
  {Artificial Intelligence Researcher}
  {eButterfly}
  {
  Led the development of a global deep learning model to automatically identify
butterfly species from images submitted by citizen scientists to the eButterfly
platform. Key contributions: \\
  \begin{contributionlist}
      \contribution{Aggregated a large-scale, diverse dataset covering over
18,000 butterfly species using GBIF-indexed data}
      \contribution{Integrated biological priors, including geographical
distribution, to enhance model performance}
      \contribution{Trained and optimized deep learning models for fine-grained
classification under a highly imbalanced long-tail distribution}
    \end{contributionlist}\\
\texttt{Computer Vision}\slashsep\texttt{Fine-grained
Classification} \slashsep\texttt {PyTorch}}\\

%   \entry
%   {7/2021 -- 3/2023\\\footnotesize{Manaus, AM, Brazil}}
%   {Undergraduate Research Mentor}
%   {IComp - UFAM}
%   {Responsible for mentoring 6 undergraduate students in Computer Science
% through their first research project using computer vision and machine learning
% under the Samsung UFAM Project for Education and Research (SUPER).\\
% \texttt{Mentoring}\slashsep\texttt{TensorFlow}\slashsep\texttt{Computer
% Vision}\slashsep\texttt{Python}}

\entry{4/2017 -- 6/2018\\\footnotesize{Manaus, AM, Brazil}}
  {Computer Vision Researcher}
  {Mamiraua Institute}
  {
    I worked as a computer vision researcher on Project Providence, focusing on
developing deep learning models for animal species recognition on the camera
vision module. My tasks included:\\
    \begin{contributionlist}
      \contribution{Analyzing, cleaning, and preparing camera trap image
datasets for model training}
      \contribution{Training and evaluating animal species classifiers using
deep convolutional neural networks}
      \contribution{Optimizing models to run efficiently on the Providence
camera vision module, based on a Raspberry Pi development board}
    \end{contributionlist}\\
\texttt{TensorFlow}\slashsep\texttt{Computer Vision}
\slashsep\texttt {Raspberry Pi}\slashsep\texttt{Camera Traps}}\\\\\\

%   \entry{3/2015 -- 3/2017\\\footnotesize{Manaus, AM, Brazil}}
%   {Computer Engineer}
%   {IATECAM}
%   {
%     I mainly worked as an embedded software engineer for projects using Altera
% FPGA technology. My responsibilities included:\\
%     \begin{contributionlist}
%       \contribution{Develop Linux device drivers for customized IP blocks}
%       \contribution{Develop embedded Linux applications}
%       \contribution{Develop computer vision models for detecting soldering
% problems}
%     \end{contributionlist}\\
%     \texttt{C}\slashsep\texttt{FPGA}\slashsep\texttt{Embedded
% Software}\slashsep\texttt{Linux}\slashsep\texttt{Computer Vision}}
%
%   \entry{11/2013 -- 3/2015\\\footnotesize{Manaus, AM, Brazil}}
%   {Software Developer}
%   {CETELI - UFAM}
%   {
%     Client: Samsung R\&D Center (SIDIA)\\
%     \begin{contributionlist}
%       \contribution{Developed apps for Android platform with focus on
% front-end:
% design of custom widgets, layouts, custom animations, etc.}
%       \contribution{Developed web projects using PHP, HTML5, CSS3, JavaScript,
% and JQuery.}
%     \end{contributionlist}\\
%
% \texttt{Java}\slashsep\texttt{PHP}\slashsep\texttt{JavaScript}\slashsep\texttt{
% Android}}
%
% \entry{2/2012 -- 8/2013\\\footnotesize{Manaus, AM, Brazil}}
%   {Embedded Software Engineer Intern}
%   {Map Innovation}
%   {
%     I worked on the development of an industrial data collection equipment.
% Tasks included::\\
%     \begin{contributionlist}
%       \contribution{Develop device drivers for embedded Linux}
%       \contribution{Customize the bootloader U-Boot}
%       \contribution{Develop firmware for ARM using FreeRTOS}
%       \contribution{Develop a complete functional testing system}
%     \end{contributionlist}\\
% \texttt{C}\slashsep\texttt{Device
% Drivers}\slashsep\texttt{Embedded Software}\slashsep\texttt
% { Embedded Linux }}
%
% \entry{4/2008 -- 1/2009\\\footnotesize{Manaus, AM, Brazil}}
%   {Electronic Engineering Intern}
%   {Terra da Amazonia Ltda.}
%   {I worked on the manufacturing test engineering team responsible for the
% motherboards production lines. My responsibilities included identifying and
% fixing problems on test stations and deploying and adapting functional tests
% for
% new products.}
\end{entrylist}

%-------------------------------------------------------------------------------
%	EDUCATION
%-------------------------------------------------------------------------------

\cvsect{Education}

\begin{entrylist}
  \entry
  {Expected 12/2025\\\footnotesize{Manaus, AM, Brazil}}
  {Ph.D. Student - Informatics}
  {Federal University of Amazonas}
  {Advisor: Eulanda Miranda | Co-Advisor: Juan Colonna\\
  Research area: Computer Vision \\

  This research aims to advance the automated analysis of camera-trap images
using computer vision and deep learning techniques. It focuses on key ecological
monitoring tasks such as filtering empty images, species classification (at
image and capture event levels), and individual counting.

  }
  \entry
  {04/2019\\\footnotesize{Manaus, AM, Brazil}}
  {Master's Degree - Informatics}
  {Federal University of Amazonas}
  {Advisor: Eulanda Miranda | Co-Advisor: Juan Colonna\\
  Research area: Computer Vision\\\\
 Thesis: Um estudo sobre abordagens para avaliação out-of-sample de modelos de
classificação de animais em imagens de armadilhas fotográficas}
  \entry
  {01/2015\\\footnotesize{Manaus, AM, Brazil}}
  {Engineer's Degree - Computer Engineering}
  {Federal University of Amazonas}
  {Embedded Software Engineering, Machine Learning}
\end{entrylist}


\cvsect{Publications}

\begin{minipage}[t]{\textwidth}
  % \vspace{-\baselineskip} % Required for vertically aligning minipages

Aditya Jain*, \textbf{Fagner Cunha}*, Michael Bunsen*, Juan Sebasti{\'a}n
Ca{\~n}as, L{\'e}onard Pasi, David Rolnick, et al. Insect identification in the
wild: The AMI dataset. In European Conference on Computer Vision (ECCV), 2024,
pp. 55-73.\\
{*\scriptsize Equal contribution}\\

Aditya Jain*, \textbf{Fagner Cunha}*, Michael Bunsen*, L{\'e}onard Pasi, Anna
Viklund, Maxim Larrivée \& David Rolnick. A machine learning pipeline for
automated insect monitoring. In NeurIPS 2023 Workshop on Tackling Climate
Change with Machine Learning. arXiv preprint arXiv:2406.13031.\\
{*\scriptsize Equal contribution}\\

\textbf{Fagner Cunha}, Eulanda M. dos Santos, \& Juan G.
Colonna. Bag of tricks for long-tail visual recognition of animal species in
camera-trap images.  Ecological Informatics, v. 76, p. 102060, 2023.\\

\textbf{Fagner Cunha}, Eulanda M. dos Santos, Raimundo Barreto, \& Juan G. 
Colonna. Filtering Empty Camera Trap Images in Embedded Systems. In Proceedings 
of the IEEE/CVF Conference on Computer Vision and Pattern 
Recognition (CVPR) Workshops, 2021, pp. 2438-2446.\\

% \textbf{Fagner Cunha}, Eulanda M. dos Santos, \& Eduardo J. P. Souto. Detection
% of Phishing Webpages Using Machine Learning Techniques. In Proceedings
% of the 12th Brazilian Symposium on Information and Computer System Security,
% 2012, pp. 491-500.\\
\end{minipage}

\cvsect{Honors \& Awards}

\begin{entrylist}
  \entry
  {06/2021}
  {1st Place in the iWildCam 2021 Competition}
  {CVPR - FGVR Workshop}
  {Count the number of animals of each species present in a sequence of images.}
  \entry
  {11/2012}
  {Best Paper Award at the Workshop of Undergraduate Research}
  {SBSEG}
  {Award received for the paper: Detection
of Phishing Webpages Using Machine Learning Techniques}
  \entry
  {08/2011}
  {Professor Abraham Moysés Cohen Award}
  {Federal University of Amazonas}
  {Best undergraduate research work in Exact Sciences at the XIX Congress of
Scientific Initiation for the work: Detection of Phishing Webpages.}
\end{entrylist}


%----------------------------------------------------------------------------------------
%	ADDITIONAL INFORMATION
%----------------------------------------------------------------------------------------

\begin{minipage}[t]{0.3\textwidth}
  \vspace{-\baselineskip} % Required for vertically aligning minipages

  \cvsect{Languages}

  \begin{skills}
    \skillset{Portuguese}{native}
    \skillset{English}{professional working proficiency}
  \end{skills}
\end{minipage}

\end{document}
